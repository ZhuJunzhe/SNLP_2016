\documentclass[a4paper]{article}
\usepackage[utf8]{inputenc}
\usepackage{amsmath}
\usepackage{a4wide}
\usepackage{amsfonts}
\usepackage{amssymb}
\usepackage{makeidx}
\usepackage{graphicx}
\usepackage{lmodern}
\usepackage{subcaption}
\usepackage{kpfonts}
\title{Statistical Natural Language Processing \\SS2016  Exercise 1}
\author{Junzhe Zhu, Xiaoyu Shen}
\begin{document}
\maketitle

\section{Text Preprocessing}


\subsection{Text Preprocessing techniques}
\begin{itemize}
\item{Stop Word Removal }:
Stop words  do  not  contribute  to  the  context  or content  of  textual  documents.While their frequency of occurrence is high.
\item{Different character cases}
: The character has different cases gives the same meaning for the tokens we are counting.

\item{ headers and footers }:
They may be in the every page and might not be the actual words.
\end{itemize}



\subsection{Tokenizer for Zipf’s Law}
The book  \textit{The adventures of Tom Sawyer by Mark Twain}, as recommended in the assignment sheet. It has three different version, English, German and Finnish.And the UTF-8 encoded plain text files are used.


\section{Zipf's law}
\end{document}